
\documentclass{report}
\usepackage[portuges]{babel}
\usepackage[utf8]{inputenc}
%\usepackage[latin1]{inputenc}
\usepackage{url}
\usepackage{graphicx}
\usepackage{alltt}
\usepackage{fancyvrb}
\usepackage{listings}
\usepackage{xcolor}
\usepackage{blindtext}
\usepackage{amsmath}
\usepackage{microtype}
\usepackage{xspace}
\lstset{
	basicstyle=\small,
	numbers=left,
	numberstyle=\tiny,
	numbersep=5pt,
	breaklines=true,
    frame=tB,
	mathescape=true,
	escapeinside={(*@}{@*)}
}

\parindent=0pt
\parskip=2pt

\def\darius{\textsf{Darius}\xspace}
\def\java{\texttt{Java}\xspace}

\title{Comunicações por Computador\\3ºano\\ Trabalho Prático Nº 2\\\textbf{Desenho e implementação de um jogo distribuído na Internet}\\TP2}
\author{André Geraldes (67673) \and Patrícia Barros (67665) \and Sandra Ferreira (67709)}
\date{\today}

\begin{document}

\maketitle

\begin{abstract}
Este relatório descreve todo o processo de desenvolvimento e decisões tomadas para a realização do segundo trabalho prático da Unidade Curricular de Comunicações por Computador.\\ O problema a resolver consiste na criação de um prótotipo de um jogo em tempo real que seja baseado num serviço de distribuição de contéudos, com recurso a comunicações TCP e UDP.
 
\end{abstract}

\tableofcontents

\chapter{Introdução} \label{intro}


\section*{Contexto}
Este trabalho foi realizado no âmbito da Unidade Curricular de Comunicações por Computador e é inserido no contexto da matéria leccionada nas aulas acerca de conexões TCP, UDP e otimização da utilização da largura de banda.
\section*{Motivação}
Em aplicações distribuidas na Internet com partilha de dados a partir duma ou várias fontes, em que o número de programas clientes dos utilizadores da aplicação é muito maior que o número de servidores disponível, é normal utilizarem-se técnicas/tecnologias que permitam a poupança da largura de banda. 
\section*{Objetivos}
Com este trabalho pretende-se gerar um pequeno protótipo dum jogo totalmente criado pelo grupo, baseado num serviço de distribuição de conteúdos com posterior interação por parte dos utilizadores e que tente otimizar a utilização da largura de banda recorrendo a técnicas implementadas.
\section*{Resultados}
O trabalho final não cumpre todos os objectivos pedidos no enunciado deste trabalho, mas o grupo esforçou-se para cumprir o maior número possivel de objectivos e com isto foi possível consolidar conhecimentos sobre UDP e datagramas.

%Enquadramento

\section{Estrutura do Relatório} \label{estrutura}
A elaboração deste documento teve por base um template para criação de relatórios científicos. \\
O relatório encontra-se então estruturado da seguinte forma: no Capítulo \ref{intro} é feita uma contextualização ao assunto tratado neste trabalho, seguindo-se uma introdução onde são apresentadas as metas de aprendizagem pretendidas.\\
Posteriormente, no Capítulo \ref{ae}, é exposto o enunciado do trabalho e a descrição do problema.\\
Imediatamente após, no Capítulo \ref{concepcao}, é apresentada a proposta de resolução do problema e exemplos de código. São também descritas algumas classes do programa e estruturas utilizadas.\\
No Capítulo \ref{testes} são apresentados os testes realizados e os seus resultados.\\
Por último, no Capítulo \ref{concl}, faz-se uma análise crítica relativa quer ao desenvolvimento do projeto quer ao seu estado final e ainda é feita uma abordagem ao trabalho futuro.

\chapter{Análise e Especificação} \label{ae}
\section{Enunciado}\label{enunciado}
Planeamento e implementação dum serviço de distribuição de conteúdos que tente otimizar a utilização da largura de banda por transmissão seletiva, fiável e com controlo de fluxo e de erros, dos conteúdos entre servidores usando unicast \textbf{TCP} (apenas para grupos de servidores que sirvam utilizadores que estão a jogar num determinado momento) e transmissão por unicast e broadcast \textbf{UDP} para todos os clientes da rede local \textbf{IP} (onde não costuma haver restrições relevantes na largura de banda). As comunicações do projeto são implementadas à custa de programação em sockets \textbf{TCP} e \textbf{UDP} e não deve utilizar nenhuma tecnologia baseada em http ou implementados interfaces baseados em serviços web.

\section{Descrição do Problema}\label{descricao}
Como descrito na secção anterior o desafio que nos foi proposto consiste na criação de uma aplicação numa linguagem de programação à nossa escolha que seja constituida por um servidor e vários clientes que jogam os jogos lançados pelo servidor. \\
Posto isto tivemos então de, primeiramente, pensar e decidir uma linguagem para a implementação e resolução do problema, e escolhemos \textbf{Java}. De seguida foi necessário decidir algumas classes e possíveis alternativas e por fim criar uma interface que permita ao jogador interagir com o servidor de uma maneira mais simples.

\chapter{Concepção/desenho da Resolução}\label{concepcao}

\section{Escolha da linguagem}\label{linguagem}
O principal princípio que seguimos aquando de decidir a linguagem a utilizar foi a simplicidade desta, visto que estavamos habituados a ela e nos permitiria programar sockets \textbf{UDP} e \textbf{TCP} sem problemas e aplicá-la de forma a que os passos seguintes do trabalho não fossem excessivamente complicados.

\section{Servidor}\label{servidor}
O servidor terá que ser uma entidade sempre activa à espera de pedidos dos seus clientes, instalado numa porta fixa para que os clientes possam contactá-lo. Para tal usamos as seguintes instruções: 
\begin{lstlisting}[language=Java]
protected DatagramSocket socket = null;
...
socket = new DatagramSocket(4445);
\end{lstlisting}
A porta utilizada será a 4445. O servidor fica em \textit{loop} a receber pedidos dos seus clientes e é feito da seguinte forma:
\begin{lstlisting}[language=Java]
byte[] buf = new byte[256];
 
DatagramPacket packet = new DatagramPacket(buf, buf.length);
socket.receive(packet);
String received = new String(packet.getData(), 0, packet.getLength());
\end{lstlisting}
Aqui definimos um datagrama para guardar o datagrama vindo do cliente. \\ Após a receção do datagrama é colocado numa \textit{String} e é feito o parsing desta, para poder responder de forma eficaz ao pedido recebido.
Exemplo de parsing de uma parte do PDU:
\begin{lstlisting}[language=Java]
String[] lb;
lb = received.split("label=");
lb = lb[1].split((","));
int label = Integer.parseInt(lb[0]);
\end{lstlisting}
Neste passo é guardado na variável \textit{label} do tipo \textit{int}, o valor da label que vinha no datagrama enviado pelo cliente. Após isto é usada uma função criada por nós para saber o tipo de PDU recebido (pode ser do tipo LOGIN, LOGOUT, etc.).

\begin{lstlisting}[language=Java]
public String splitPDU(String pdu){
    String[] nova;
    nova = pdu.split("tipo=");
    nova = nova[1].split("]");
        
    return nova[0];
}
...
String op = splitPDU(received);
if(op.equals("LOGIN")){ 
    //executar accoes de login
\end{lstlisting}
Esta função faz o parse do PDU e retorna apenas o campo após \textbf{tipo=} que poderá ser por exemplo LOGIN. Após isso o servidor entra num \textit{loop} de \textit{if} e \textit{else} para tratar cada tipo de pedido e são realizadas instruções por parte do servidor para responder ao pedido recebido.

\begin{lstlisting}[language=Java]
resposta = "PDU[ver=0,seg=0,label=" + label + ",tipo=REPLY]Lista de Campos:[255=\"Utilizador ja online\"]"; 
\end{lstlisting}
Exemplo de resposta quando o utilizador que tenta fazer login já se encontra logado noutro cliente. Após a saída do tratamento do pedido e já com a resposta na variável \textit{resposta} do tipo \textit{String} é feito o envio para o cliente.
\begin{lstlisting}[language=Java]
buf = resposta.getBytes();

InetAddress address = packet.getAddress();
int port = packet.getPort(); 
packet = new DatagramPacket(buf, buf.length, address, port);
socket.send(packet); 
...
socket.close();
\end{lstlisting}

Variavéis de classe do servidor.
\begin{lstlisting}[language=Java]
private HashMap<String,Utilizador> utilizadores;
private HashMap<Integer,Pergunta> perguntas;
private HashMap<String,Desafio> desafios;
private String userOnline;
private String musicDir;
private String imagDir;
private int nrPerguntas;
protected DatagramSocket socket = null;
\end{lstlisting}
É nestas variavéis que ficam guardados os utilizadores registados no sistema, todas as perguntas existentes e os desafios criados pelos jogadores.
Na inicialização do servidor temos
\begin{lstlisting}[language=Java]
this.utilizadores = new HashMap<>();
this.desafios = new HashMap<>();
this.perguntas = new HashMap<>();
Utilizador ut = new Utilizador(0, "admin", "admin", "admin");
this.getUtilizadores().put("admin", ut);
this.lerPerguntas();
\end{lstlisting}
Que é onde são inicializadas as estruturas para guardar os dados, criação de um utilizador que é o \textit{admin} e a leitura das perguntas existentes de um ficheiro \textit{perguntas.txt}.

\section{Cliente}\label{cliente}
Com o servidor criado a receber pedidos e tratar destes, é necessário um cliente para fazer as trocas de pedidos e respostas. Para isto utilizamos uma interface gráfica baseada em \textbf{Swing}. A maneira como o cliente manda pedidos e recebe respostas é idêntica à do servidor como mostramos a seguir.
\begin{lstlisting}[language=Java]
DatagramSocket socket = new DatagramSocket()) {
byte[] buf = new byte[256]; <- Usado no envio do pedido
byte[] rec = new byte[256]; <- Usado na rececao do pedido
InetAddress address = InetAddress.getLocalHost();

String pedido = "PDU[ver=0,seg=0,label=" + label++ + ",tipo=LIST_RANKING]Lista de campos[]"; 

buf = pedido.getBytes();
DatagramPacket packet = new DatagramPacket(buf, buf.length, address, 4445);
socket.send(packet);

packet = new DatagramPacket(rec, rec.length);
socket.receive(packet);
String received = new String(packet.getData(), 0, packet.getLength());
\end{lstlisting}
Para os outros casos é idêntico, só alterando algumas coisas e adaptando para cada tipo de pedido.

\section{Resultados}\label{resultados}
Nesta secção vamos apresentar algumas imagens dos resultados obtidos.
\begin{center}
\includegraphics[scale=0.7]{}
\label{fig1}
\end{center}

\chapter{Conclusão} \label{concl}
Este trabalho é uma aplicação que contém um servidor e um ou mais clientes que comunicam entre si por sockets UDP, não foram implementadas comunicações por TCP nem multiservidores.
\\Para nós foi uma boa oportunidade para aprender mais sobre trocas de mensagens por sockets UDP e tratamento das mesmas.
\\O grupo não conseguiu implementar algumas funcionalidades pedidas como por exemplo jogar um desafio.
\end{document} 